\chapter{Conclusions and Future Work}\label{conclusions:chap}

In this chapter, we summarize the contributions of this dissertation and discuss possible future research directions. 

\section{Summary}
The goal of this dissertation is to explore geometrical approaches that enable efficient wireless service in large scale networks. These approaches identify key characteristics of the underlying geometry to better manage the radio resources in different network settings.
%

%We have explored non-convex optimization in the context of wireless communications under geometrical perspectives, using new techniques developed for seemingly unrelated problems. We have also proposed new geometrical perspectives for both grant-free access and grant-free scheduling, leveraging the underlying properties of the computational space to improve the performance of these tasks with reasonable computational complexity and minimal ad-hoc settings.


In Chapter~\ref{system:chap}, we lay the foundation for a general system model that allows the study of massive wireless networks with a multi-antenna base station (BS) and multiple single-antenna devices, in both uplink (MAC) and downlink (BC) scenarios. This model is then developed for two distinct operating regimes, depending on the activity level of the devices. 
%of the  whether the BS is aware of the active users or not. 

In the case of low-activity devices, with sporadic and non-timed links, an unknown number of devices transmit to the BS in uplink operation at any given time. To service the devices, the BS attempts to provide \emph{access control}, without knowing which devices are active.
Due to the limitations of traditional approaches for providing access to those users, we propose the adoption of grant-free access via use blind signal recovery methods, and in particular, using the Constant Modulus Algorithm (CMA). In the first part of this dissertation (Chapters~\ref{wfcma:chap} and~\ref{rocma:chap}), we formulate two CMA-based approaches for blind signal recovery under different optimization perspectives.
%, and we analyze the geometry of each formulation to develop convergence guarantees. 

In Chapter~\ref{wfcma:chap}, we develop a regularized CMA-based cost function for multiple source recovery. The regularization term forces recovered signals to be uncorrelated, recovering distinct sources. To solve the optimization problem, we adopt the promising Wirtinger Flow (WF) algorithm. As WF provides theoretical guarantees for similar nonconvex problems under mild conditions and limited data samples, we leverage the convergence analysis of WF and generalize to consider the statistical characteristics of the signals in CMA. By characterizing the local geometry of CMA, we obtain convergence guarantees for CMA in the finite-sample scenario and general QAM modulations, both for simple and multiple source recovery. Furthermore, these results facilitate the selection of a more aggressive stepsize than commonly used in traditional gradient-descent methods, tackling slow convergence with no significant increase in computational cost.

In Chapter~\ref{rocma:chap} we dismiss regularization and instead opt for a constrained optimization of CMA-based multiple source recovery, where the search space forces multi-lateral orthogonality of recovered signals. We characterize and redefine the constraint set as a Riemannian manifold, which we further develop to remove search directions corresponding to rotations of the demixers. This steps ensures that the cost function, restricted to the manifold, has a positive definite Riemannian Hessian in optimum solutions. Using this fact, we provide global convergence guarantees with high probability and limited data samples, regardless of initialization or parameter tuning. Numerical tests show successful recovery of all detected sources with a reasonable number of samples, for practical system sizes and different modulation schemes. Furthermore, the proposed Riemannian approach offers a good tradeoff between computation complexity and interference suppression. 


%\section{Unsupervised Riemannian User Scheduling}

In the second part of this dissertation, we turn to the problem of servicing a large number of very active users known to the base station, both in uplink and downlink scenarios. In this operation mode, the base station needs to attempt efficient \emph{user scheduling} to service most or all users with reasonable quality of service and using limited resources, i.e., it needs to allocate users in groups with low co-channel interference (CCI).
In Chapter~\ref{usch:chap}, we investigate the user scheduling problem by exploiting spatial diversity in MIMO systems. To mitigate CCI, instead of directly scheduling users with high channel dissimilarity, we propose a new unsupervised learning paradigm for MIMO user scheduling. This two-step strategy leverages the strengths of unsupervised learning with domain knowledge for the specific characterization of spatial diversity for low CCI. In a first step, we identify users whose channel state information (CSI) are highly similar in the sense of spatial correlation. This is possible by clustering CSI vectors in the Grassmannian manifold, which encodes subspace span in the geometry itself. After clustering, a greedy scheduling scheme exploits learning outcomes and determines user groups with low CCI. Our numerical test indicate that a combination of  learning-enabled channel clustering can exploit the spatial compatibility effectively to reduce inter-group interference when compared with other benchmarks. Furthermore, this paradigm is generalizable to different learning schemes and scheduling metrics.

\section{Extensions}


\begin{itemize}
	\item \textbf{Stronger convergence guarantees for Blind Signal Recovery:} 
	Our convergence guarantees are formulated in noiseless scenarios. Reformulations of WF in phase retrieval have shown robustness against arbitrary corruptions \cite{Zhang2016mediantruncatedwf} or random noise \cite{Cai2016thresholdedwf,Lazreg2018optimasparselrobustwf} under mild assumptions, and these results could be leveraged in the context of blind signal recovery. Another interesting possibility is to consider stronger initialization methods with provable convergence improvement, and the study of our proposed methods in the more practical scenario of blind recovery under MIMO ISI fading channels
	
	\item \textbf{Exploiting additional information in Blind Signal Recovery:} 
	Practical wireless systems consist of sophisticated protocols, and even in the case of IoT deployments, the transceivers will commonly have access to more information than just the received (incoming) signal samples. For example, the devices will use forward error correction codes and channel coding, and the algorithm could verify if the recovered signal belongs to the set of valid codewords. Signals can also change constellation size over the duration of the data packet, e.g. WiFi, which has QPSK preambles and $M$-QAM payloads. Overall, our proposed signal recovery process could be enhanced if we consider and exploit these characteristics, be it by modifying the underlying geometry, leverage other non-convex optimization techniques such as proximal gradients, among other options.
	
	\item \textbf{Complexity reduction for Blind Signal Recovery:}
	Different families of cost functions have been proposed for blind equalization and beamforming \cite{Benveniste1980bgr,Sato1975,Shalvi1990}, which are either nonconvex or nonsmooth. Moreover, similar cost functions have been studied in WF-like extensions for phase retrieval \cite{Zhang2016mediantruncatedwf,Cai2016thresholdedwf,Lazreg2018optimasparselrobustwf}, which enjoy reduced computational and iteration complexity. These recent works use different approaches for their analysis, which could be leveraged to improve our theoretical results. Nonconvex and non-smooth cost functions can also be adopted in Riemannian optimization \cite{Kovnatsky2016madmmriemann,Absil2019nonsmoothriemannian}. Additionally, the proposed algorithms could be reformulated to consider stochastic or mini-batch implementations, further reducing computational complexity without incurring in performance loss.
	
	\item \textbf{Generalizations of MIMO User Scheduling:} 
	Our results of unsupervised user scheduling are promising, although the system model is rather simple and could include common practical considerations. One clear example is systems with both time- and frequency-division multiple access (TDMA/FDMA), which are now commonplace in commercial applications. New geometries can leverage the similarity of users in multiple bands within a time slot, providing further insights to be considered in the scheduling problem. Other example is to consider users with different rate requirements that could utilize multiple resource blocks in a given frame, and how to leverage CSI similarity with unequal rates. Moreover, the design of similarity-assisted scheduling algorithms can further exploit additional information and consider other performance metrics such as user data rate and fairness.
\end{itemize}
