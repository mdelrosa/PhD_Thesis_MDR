\chapter{Compressed Sensing}
\label{appdx:compressed-sensing}

In Section~\ref{sect:hetero-markov} of Chapter~\ref{chap:p2d}, we introduced a heterogeneous differential encoder which utilized deep compressed sensing networks. This Appendix provides a brief overview of the relevant theory from compressed sensing based on the excellent survey \cite{ref:Marques2019ReviewOfSparseRecovery}.

Given a signal $\mathbf{x}\in\mathbb{R}^N$, denote a random measurement of this signal as 
\begin{align*}
    \mathbf{y} = \mathbf{\Phi}\mathbf{x}
\end{align*}
where $\mathbf{\Phi}\in\mathbb{R}^{M\times N}$ is referred to as the \emph{measurement matrix} and $\mathbf{y}\in\mathbb{R}^{M}$ is a low-dimensional measurement (i.e., $M << N$). The goal of compressed sensing is to recover the original signal $\mathbf{x}$ based on the low-dimensional measurement $\mathbf{y}$. The least-squares solution to this problem is given as
\begin{align}
    \min_{\hat{\mathbf{x}}}\|\hat{\mathbf{x}}\|_2 \; \text{subject to} \; \|\mathbf{y} - \mathbf{\Phi}\hat{\mathbf{x}}\|_2^2 < \epsilon \label{eq:undet-ls}
\end{align}
where $\epsilon > 0$ is some error tolerance. By construction, the matrix $\Phi$ has more columns than rows (i.e., $N >> M$), and consequently, the linear system (\ref{eq:undet-ls}) underdetermined. Furthermore, the least-squares solution cannot return a sparse vector, and instead, the recovery of $\mathbf{x}$ is typically framed as a sparsity-constrained least-squares estimation problem, i.e.
\begin{align}
    \min_{\hat{\mathbf{x}}}\|\hat{\mathbf{x}}\|_0 \; \text{subject to} \; \|\mathbf{y} - \mathbf{\Phi}\hat{\mathbf{x}}\|_2^2 < \epsilon \label{eq:sparse-ls}
\end{align}
Under certain constraints, the original signal $\mathbf{x}$ can be perfectly reconstructed. However, this perfect reconstruction requires a combinatoric search over $\begin{pmatrix}N \\ s\end{pmatrix}$ (where $s$ is the number of non-zero elements in $\mathbf{x}$).
% \begin{align*}
%     \argmin_{\hat{\mathbf{x}}} \|\mathbf{y}-\mathbf{\Phi}\hat{\mathbf{x}}\|_2^2 + \lambda|\mathbf{\Phi}\mathbf{x}|_j.
% \end{align*}
Instead, the problem is often relaxed to use the $\ell_1$ norm,
\begin{align}
    \min_{\hat{\mathbf{x}}}\|\hat{\mathbf{x}}\|_1 \; \text{subject to} \; \|\mathbf{y} - \mathbf{\Phi}\hat{\mathbf{x}}\|_2^2 < \epsilon, \label{eq:lasso-ls}
\end{align}
which is referred to as the least absolute shrinkage and selection operation (LASSO). 

\section{The ISTA algorithm}
To solve the LASSO, it is possible to use proximal gradient methods from convex optimization \cite{ref:beck2009fast}. One such method is the iterative shrinkage threshold algorithm (ISTA), which we will explain in this section. Generally, gradient-based methods solve the problem of the form,
\begin{align*}
    \min \left\{f(\mathbf{x})+\lambda\|\mathbf{x}\|_1\right\},
\end{align*}
using the iterative steps,
\begin{align*}
    \mathbf{x}_k &= \argmin_{\mathbf{x}} \left\{f(\mathbf{x}_{k-1}) + \langle \mathbf{x} - \mathbf{x}_{k-1}, \nabla f(\mathbf{x}_{k-1}) \rangle + \frac{1}{2t_k}\|\mathbf{x} - \mathbf{x}_{k-1}\|^2 + \lambda\|\mathbf{x}\|_1\right\} \\
    &= \argmin_{\mathbf{x}} \left\{\frac{1}{2t_k}\|\mathbf{x} - (\mathbf{x}_{k-1} - t_k\nabla f(\mathbf{x}_{k-1}))\|^2 + \lambda\|\mathbf{x}\|_1\right\},
\end{align*}
where $t_k>0$ is the stepsize for the algorithm. Note that the second line is admitted by ignoring the constant terms. The $\ell_1$ term is separable, and consequently, computing the iteration $\mathbf{x}_k$ can be done by solving a one-dimensional minimization problem for each component of $\mathbf{x}_k$,
\begin{align*}
    \mathbf{x}_k &= \mathcal{T}_{\lambda t_k} (\mathbf{x}_{k-1} - t_k\nabla f(\mathbf{x}_{k-1})).
\end{align*}
Note the shrinkage operator, $\mathcal{T}_{\alpha} : \mathbb{R}^n \to \mathbb{R}^n$,
\begin{align*}
    \mathcal{T}_{\alpha}(\mathbf{x})_i &= (|x_i| - \alpha)_{+}(x_i)
\end{align*}
The 