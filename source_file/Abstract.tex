\newpage
{\baselineskip 14pt \rightline{Mason del Rosario} \rightline{September,
\the\year} \rightline{Electrical and Computer Engineering}} \vspace{27pt}
\centerline{\Huge \bf {Abstract}} \vspace{18pt}
\addcontentsline{toc}{section}{Abstract}

Future wireless communications networks will rely heavily on massive MIMO technologies where a base station (BS) with large multiantenna arrays serve a large number of user equipment (UE) terminals. Such multiantenna arrays enable high capacity communications via beamforming, as evidenced by work in information theory \cite{ref:goldsmith2003capacity}. To achieve capacity in massive MIMO networks, the base station requires accurate estimates of the channel state information (CSI) in order to precode (decode) transmitted (received) messages \cite{ref:marzetta2016fundamentals}.

CSI can be acquired using pilot signals, and in time-division duplex (TDD) mode, channel reciprocity allows the BS to estimate the downlink CSI via pilots in uplink transmissions. However, in frequency division duplex (FDD) mode, channel reciprocity between uplink and downlink channels is weak, and the BS must rely on feedback from the UE to estimate downlink CSI. Specifying an appropriate CSI feedback scheme is a key issue and involves a reducing feedback bandwidth while maintaining accurate downlink CSI estimates.

Conventional methods for CSI feedback compression typically rely on compressed sensing (CS), which seeks to reconstruct high-dimensional data based on low-dimensional measurements (see \cite{ref:Marques2019ReviewOfSparseRecovery} for a survey of CS methods). Many CS methods rely on convex relaxations of an underdetermined least-squares problem, and such methods rely on iterative solvers (e.g., the proximal gradient method). When using an iterative solver, reconstruction can consume an undue amount of time even when measurements are available, making faster methods for reconstruction desirable.

Recent work in deep learning for compressed CSI estimation has presented viable alternatives to CS methods \cite{ref:csinet}. Such work typically employs convolutional neural networks (CNNs) to learn compressed representations of high-dimensional CSI, and the architectures used in CNN-based works can be placed in one of two categories. The first category, CNN-based autoencoders, consists of networks which utilize two subnetworks: an encoder network which learns a low-dimensional representation with the original data as an input and a decoder network which estimates the original data as a function of the low-dimensional representation. The second category, unrolled optimization networks, draws inspiration from the aforementioned CS methods by structuring the CNN as a finite number of repeated blocks, each of which imitates an iteration of a given CS algorithm.

This thesis explores both CNN autoencoders and unrolled optimization networks for CSI estimation while focusing domain knowledge to improve the performance of these CSI estimation networks with respect to accuracy, feedback rate, or network efficiency. Prior works have demonstrated superior performance over these architectures over conventional CS methods \cite{ref:csinet,ref:Guo2022FISTANet}, and this thesis investigates the myriad ways that domain knowledge in wireless channels and communications protocols can be leveraged to improve CSI estimation.

% Other points to touch on
% - Accounting for pilots? P2D
% - Bridging gap b/n compressed sensing and deep learning?
%   - Why do autoencoders underperform deep CS methods? Does any lit support this (yes in CV)?