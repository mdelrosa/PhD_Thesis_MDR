\newpage
% {\baselineskip 14pt \rightline{Mason del Rosario} \rightline{September,
% \the\year} \rightline{Electrical and Computer Engineering}} \vspace{27pt}
\chapter*{\centering Abstract}
\addcontentsline{toc}{chapter}{Abstract}

Future wireless communications networks will implement massive MIMO technologies to significantly improve spectrum efficiency where a base station (BS) with large multiantenna arrays serve a large number of user equipment (UE) terminals. Such multiantenna arrays enable high capacity communications via beamforming, as evidenced by work in information theory. To achieve the expected capacity in massive MIMO networks, the base station requires accurate estimates of downlink channel state information (CSI) at the transmitter for precoding.

Typically, receivers can estimate CSI using known pilot signals. In the special case of time-division duplex (TDD) mode, channel reciprocity allows the BS to estimate the downlink CSI via pilots in uplink transmissions. However, in frequency division duplex (FDD) mode, channel reciprocity between uplink and downlink channels is comparatively weak, and the BS must also rely on feedback from the UE to estimate downlink CSI. Specifying an appropriate CSI feedback scheme is a key issue and involves reducing feedback bandwidth while maintaining accurate downlink CSI estimates.

Conventional methods for CSI feedback compression can rely on compressive sensing (CS), which seeks to reconstruct high-dimensional data based on low-dimensional measurements. Many CS methods are based on convex relaxations of an underdetermined least-squares problem, and such methods rely on iterative solvers (e.g., the proximal gradient method). When using an iterative solver, reconstruction can consume an undue amount of time even when measurements are available, making faster CSI reconstruction methods an attractive research direction.

Recent works in deep learning for compressed CSI estimation have presented viable alternatives to CS methods. These proposed methods typically employ convolutional neural networks (CNNs) to extract compressed representations of high-dimensional CSI. Deep learning architectures used in CNN-based works can be placed in one of two categories. The first category is CNN-based autoencoders, networks which utilize two subnetworks: an encoder network which learns a low-dimensional representation with the original data as an input and a decoder network which estimates the original data with the encoder's low-dimensional representation as an input. The second category is unrolled optimization networks, inspired by the aforementioned CS methods by structuring the CNN as a finite number of repeated blocks with each block imitating an iteration of a given CS algorithm.

This dissertation explores both CNN autoencoders and unrolled optimization networks for CSI estimation. However, instead of taking a black box approach, we focus on domain knowledge, including physical insight into the wireless channel or features of the communications protocol, to improve the performance of these CSI estimation networks with respect to accuracy, feedback rate, or network efficiency. Prior works have demonstrated superior performance of these architectures over conventional CS methods, and this dissertation investigates the myriad ways that domain knowledge of wireless channels and communications protocols can be leveraged to improve CSI estimation.

The first research direction discussed is spherical normalization, which involves scaling CSI data by the channel's power and can improve CSI estimation accuracy without increasing model complexity. The second research innovation applies deep differential encoding, which estimates the error in CSI estimates based on a one-step Markov model. We describe a deep differential encoder which offers a significant reduction in computational complexity when compared to state-of-the-art recurrent neural networks while achieving superior estimation accuracy. In the third research direction, we use the practical, sparse frequency domain pilots to estimate the truncated delay domain, enabling the usage of the delay domain CSI commonly adopted in deep learning based CSI estimation works. Additionally, we implement a heterogeneous differential encoder, which uses different network architectures at each timeslot and offers superior estimation accuracy over the prior homogeneous differential encoders. In the final chapter, we propose model reuse, where a smaller model is used multiple times on a relatively large input, and we show that this method can maintain comparable estimation accuracy while reducing computational complexity.