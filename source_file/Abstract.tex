\newpage
% {\baselineskip 14pt \rightline{Mason del Rosario} \rightline{September,
% \the\year} \rightline{Electrical and Computer Engineering}} \vspace{27pt}
\centerline{\Huge \bf {Abstract}} \vspace{18pt}
\addcontentsline{toc}{section}{Abstract}

Future wireless communications networks will implement massive MIMO technologies where a base station (BS) with large multiantenna arrays serve a large number of user equipment (UE) terminals. Such multiantenna arrays enable high capacity communications via beamforming, as evidenced by work in information theory. To achieve capacity in massive MIMO networks, the base station requires accurate estimates of downlink channel state information (CSI) in order to precode (decode) transmitted (received) messages.

CSI can be estimated using known pilot signals. In the special case of time-division duplex (TDD) mode, channel reciprocity allows the BS to estimate the downlink CSI via pilots in uplink transmissions. However, in frequency division duplex (FDD) mode, channel reciprocity between uplink and downlink channels is weak, and the BS must also rely on feedback from the UE to estimate downlink CSI. Specifying an appropriate CSI feedback scheme is a key issue and involves reducing feedback bandwidth while maintaining accurate downlink CSI estimates.

Conventional methods for CSI feedback compression typically rely on compressive sensing (CS), which seeks to reconstruct high-dimensional data based on low-dimensional measurements (see for a survey of CS methods). Many CS methods are based on convex relaxations of an underdetermined least-squares problem, and such methods rely on iterative solvers (e.g., the proximal gradient method). When using an iterative solver, reconstruction can consume an undue amount of time even when measurements are available, making faster methods for reconstruction desirable.

Recent works in deep learning for compressed CSI estimation has presented viable alternatives to CS methods. These proposed methods typically employ convolutional neural networks (CNNs) to learn compressed representations of high-dimensional CSI. Deep learning architectures used in CNN-based works can be placed in one of two categories. The first category is CNN-based autoencoders, networks which utilize two subnetworks: an encoder network which learns a low-dimensional representation with the original data as an input and a decoder network which estimates the original data as a function of the low-dimensional representation. The second category is unrolled optimization networks, which draw inspiration from the aforementioned CS methods by structuring the CNN as a finite number of repeated blocks with each block imitating an iteration of a given CS algorithm.

This dissertation explores both CNN autoencoders and unrolled optimization networks for CSI estimation while focusing on domain knowledge, including physical insight into the wireless channel or features of the communications protocol, to improve the performance of these CSI estimation networks with respect to accuracy, feedback rate, or network efficiency. Prior works have demonstrated superior performance of these architectures over conventional CS methods, and this dissertation investigates the myriad ways that domain knowledge of wireless channels and communications protocols can be leveraged to improve CSI estimation.

% Other points to touch on
% - Accounting for pilots? P2D
% - Bridging gap b/n compressive sensing and deep learning?
%   - Why do autoencoders underperform deep CS methods? Does any lit support this (yes in CV)?