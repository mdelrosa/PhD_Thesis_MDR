\chapter{Conclusion} \label{chap:conclusion}

This thesis surveyed techniques to improve the performance and efficiency of deep neural networks for the task of MIMO CSI estimation. In Chapter~\ref{chap:sph_norm} we discussed the importance of data pre-processing techniques, and we showed the efficacy of our proposed spherical normalization technique. In Chapter~\ref{chap:markovnet}, we exploited the temporal correlation of the wireless channel, and we demonstrated the superior performance and efficiency of a deep differential encoder compared to recurrent neural networks. Finally, in Chapter~\ref{chap:p2d}, we presented two main contributions: an accurate estimator of the delay domain CSI based on sparse frequency domain pilots and a hetergeneous differential encoding network combining deep compressed sensing networks with autoencoders. We showed the accuracy of our pilot-based delay domain estimator, even under aggressive sparsity and noisy pilot estimates. Furthermore, we verified the improved performance of heterogeneous networks over homogeneous networks.

Across all these works, we investigated techniques which exploited domain knowledge of the wireless channel to improve estimation accuracy and computational efficiency while better conforming to 3GPP protocols. Further work in CSI compression should take a similar approach by taking advantage of features of the wireless channel, the communications protocol, or CSI data itself.